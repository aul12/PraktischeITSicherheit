\documentclass[DIN, pagenumber=false, fontsize=11pt, parskip=half]{scrartcl}

\usepackage{amsmath}
\usepackage{amsfonts}
\usepackage{amssymb}
\usepackage{enumitem}
\usepackage[utf8]{inputenc} % this is needed for umlauts
\usepackage[T1]{fontenc} 
\usepackage{commath}
\usepackage{xcolor}
\usepackage{booktabs}
\usepackage{float}
\usepackage{tikz-timing}
\usepackage{tikz}
\usepackage{multirow}
\usepackage{colortbl}
\usepackage{xstring}
\usepackage{circuitikz}
\usepackage{listings} % needed for the inclusion of source code
\usepackage[final]{pdfpages}
\usepackage{subcaption}
\usepackage{import}
\usepackage{bm}
\usepackage{hyperref}
\usepackage{todonotes}

\usetikzlibrary{calc,shapes.multipart,chains,arrows,shadows}

\newcommand*\keystroke[1]{%
  \tikz[baseline=(key.base)]
    \node[%
      draw,
      fill=white,
      drop shadow={shadow xshift=0.25ex,shadow yshift=-0.25ex,fill=black,opacity=0.75},
      rectangle,
      rounded corners=2pt,
      inner sep=1pt,
      line width=0.5pt,
      font=\scriptsize\sffamily
    ](key) {#1\strut}
  ;
}


%Inkscape fuckery
\newcommand{\incfig}[2][\columnwidth]{%
    \def\svgwidth{#1}
    \import{./}{#2.eps_tex}
}

\title{Intel SGX - Exercise Concept}
\author{Tim Luchterhand, Björn Petersen, Paul Nykiel}

\begin{document}
    \maketitle
    \section{Structure / Objectives}
    The exercise is structured in four parts: in the first part the motiviation for OpenSGX is given and OpenSGX
    itself is introduced. This part requires the students to read the text and get the required understanding of
    OpenSGX for the other exercises.

    The second part consists of a guide on how to setup and test the development environment for OpenSGX. This is the
    first part in which the students have to get active by themself by following the guide. This part is the base
    for the following exercises in which the students use the VM for solvin the exercises.

    In the third part the students need to implement a simple the main part of the voting software, the server
    application running in an OpenSGX-Enclave and the voting application running in normal user space.
    For this exercise the students need to understand the concept of enclaves and which parts of the application
    need to be protected by the enclave. To help the students some pointers about the structure of the application
    are given in the beginning, but students are encouraged to think about the structure and the security of the
    voting procedure through the provided questions. After the third exercise the students should be familiar with
    the concept of enclaves and how to structure an application using (Open) SGX.

    The last part of the exercise builds upon the third part by adding a local/intra attestation to their application.
    This part is designed more freely to allow the students to actively think about a secure way to run the attestation.
    Additionally it encourages the students to study OpenSGX by requiring them to read the OpenSGX examples.
    After the last exercise the students should understand the idea of attestation in the context of SGX.
    
    \section{Decisions}
    \subsection{Simplification}
    Due to the limited time available for the exercise some parts of the challenge of
    E-Voting have been simplified to concentrate on the parts which are solved by SGX.
    The simplifications are mentioned in the exercise sheet as well.

    The first simplification is that the aspect of authorization is not (securely)
    handled by the proposed application. As authorization is not a feature provided
    by SGX, but needs to be implemented through different mechanisms, we decided
    to replace it by a simple number based scheme which can be quickly implemented
    by the students.

    The next simplification for the application is the communication infrastructure:
    the proposed structure relies on named UNIX pipes (FIFOs). They have two obvious
    limitations: all voters need to use the same computer and the communication
    between the enclave and the vote application is not secure, i.e. the OS can
    still manipulate the data. But the FIFOs are sufficient to demonstrate the distributed
    implementation of the voting application, and neither network connectivity nor
    encryption are features which are directly support by OpenSGX, thus they would have
    to be implemented by the user (or by using a library of course).

    The last limitation is the usage of local attestation instead of remote attestation,
    this is only possible as the all parts of the application run on the same machine.
    Remote attestation requires a good understanding of the sigma protocol and two 
    additional applications. For local attestation only one additional application
    is required. Additional the cryptographic concepts are limited to signing and
    the challenge-response scheme. Thus we decided to use the simpler of both methods
    to fit the aspect of attestation into the exercise as well.

    \subsection{Technical}
    The exercise relies on different programs. To enable all students to be able
    to do the exercise as easy as possible we decided to use widely available,
    free software (free as in beer and FOSS).

    As not all students have access to a computer with a recent Intel processor
    we decided to use OpenSGX instead of Intel-SGX. Additionally OpenSGX can be
    used without limitations, in contrast to Intel SGX which requires a license
    from Intel.
    
    For virtual machine we decided to use Virtualbox, as the application is
    available on all major operating systems. The OS running in the virtual
    machine is Ubuntu 14.04 as this is the latest (LTS) version supported by
    OpenSGX. We tried to run it on later versions as well but they do not support
    OpenSGX out of the box.
\end{document}
