\documentclass[DIN, pagenumber=false, fontsize=11pt, parskip=half]{scrartcl}

\usepackage{amsmath}
\usepackage{amsfonts}
\usepackage{amssymb}
\usepackage{enumitem}
\usepackage[utf8]{inputenc} % this is needed for umlauts
\usepackage[T1]{fontenc} 
\usepackage{commath}
\usepackage{xcolor}
\usepackage{booktabs}
\usepackage{float}
\usepackage{tikz-timing}
\usepackage{tikz}
\usepackage{multirow}
\usepackage{colortbl}
\usepackage{xstring}
\usepackage{circuitikz}
\usepackage{listings} % needed for the inclusion of source code
\usepackage[final]{pdfpages}
\usepackage{subcaption}
\usepackage{import}
\usepackage{bm}
\usepackage{hyperref}
\usepackage{todonotes}

\usetikzlibrary{calc,shapes.multipart,chains,arrows,shadows}

\newcommand*\keystroke[1]{%
  \tikz[baseline=(key.base)]
    \node[%
      draw,
      fill=white,
      drop shadow={shadow xshift=0.25ex,shadow yshift=-0.25ex,fill=black,opacity=0.75},
      rectangle,
      rounded corners=2pt,
      inner sep=1pt,
      line width=0.5pt,
      font=\scriptsize\sffamily
    ](key) {#1\strut}
  ;
}


%Inkscape fuckery
\newcommand{\incfig}[2][\columnwidth]{%
    \def\svgwidth{#1}
    \import{./}{#2.eps_tex}
}

\title{Intel SGX}
\author{Tim Luchterhand, Björn Petersen, Paul Nykiel}

\begin{document}
    \maketitle
    \section{OpenSGX}
    Intel-SGX utilises special instructions and special CPU features to guarantee the security of an enclave.
    Thus the processor requires hardware support for Intel-SGX, which is only part of more recent Intel
    processors (6th generation and later). To be able to test SGX applications on devices without an recent
    Intel processor, there is OpenSGX. OpenSGX is a emulator (based on QEMU), that emulates the features
    provided by Intel-SGX. It was developed by the Georgia Institute of Techology to be able to not only
    test SGX enabled application, but also to be able to better debug software running in enclaves.

    As OpenSGX is purely software based it can not make the same security guarantees as Intel-SGX, in the end
    the memory of the emulator is part of the normal (not encrypted) memory, and can thus be read by the
    underlying OS.
    For our exercise this is sufficient, as we would only like to develop an application utilising SGX to demonstrate
    what is possible using SGX.

    \section{Setting up your development environment}
    OpenSGX only supports a very limited number of operating systems, this is why we decided to provide you with
    an preconfigured virtual machine image to remove the hassle of compiling OpenSGX from source. To use the image
    you need to install virtual box\footnote{\url{https://www.virtualbox.org/}}  which is available for free
    (free as in beer and free as in freedom) for all major operating systems. 

    After installing OpenSGX download our virtual box image from \todo{Link, geht das über Moodle? Eigner FTP?}.
    Now you need to create a new virtual machine from our disk image, for this follow the steps below:
    \begin{enumerate}
        \item In Virtualbox click on New (or use the shortcut \keystroke{Ctrl}+\keystroke{N})
        \item Fill in the fields of the dialog:
            \begin{itemize}
                \item \textit{Name}: Select any name for your virtual machine
                \item \textit{Machine Folder}: Select any folder on your computer which virtual box will use to save
                    files
                \item \textit{Type}: Select "Linux"
                \item \textit{Version}: Select "Ubuntu (64-bit)" (the image is based on Ubuntu 14.04, this is
                    the most recent version of Ubuntu supported by OpenSGX)
            \end{itemize}
        \item Click on "Next" to get to the "Memory Size" page
        \item Select the amount of RAM you would like to give to the VM, 1GB is sufficient, but you can give more
            if you like to
        \item Click on "Next" to get to the "Hard disk" page
        \item Select "Use an existing virtual hard disk file" and select the file that you have downloaded
        \item Click on "Create" to finish the wizard, then click on "Start" to launch the VM
        \item After booting you get a login prompt, both the user and the password is "sgx"
    \end{enumerate}

    \section{Testing your development environment}
    After starting the VM and logging in you should get familiar with the environment, for this there is a file
    called \texttt{test.c} in your home directory. Inspect the contents of the file with an editor of your choice
    (\texttt{nano} and \texttt{vim} are already installed in the VM, but you can install any other editor).

    Next compile and run the code for this run \texttt{make test.enc}, this generates the key for signing,
    builds the application, signs the application and runs it an enclave using OpenSGX.
    OpenSGX prints a lot of debug information while running, but you should also be able to see the "Hello World!"
    message printed by the test program.

    \section{Challenges E-Voting}

    \section{E-Voting using SGX}

    \section{Implementing an E-Voting application}
\end{document}
