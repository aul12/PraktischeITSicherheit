\documentclass[DIN, pagenumber=false, fontsize=11pt, parskip=half]{scrartcl}

\usepackage{amsmath}
\usepackage{amsfonts}
\usepackage{amssymb}
\usepackage{enumitem}
\usepackage[utf8]{inputenc}
\usepackage[ngerman]{babel}
\usepackage[T1]{fontenc} 
\usepackage{commath}
\usepackage{xcolor}
\usepackage{booktabs}
\usepackage{float}
\usepackage{tikz-timing}
\usepackage{tikz}
\usepackage{multirow}
\usepackage{colortbl}
\usepackage{xstring}
\usepackage{circuitikz}
\usepackage{listings}
\usepackage[final]{pdfpages}
\usepackage{subcaption}
\usepackage{import}
\usepackage{cleveref}
\usepackage{bm}


\title{Research Intel SGX}
\author{Tim Luchterhand}

\begin{document}
    \maketitle
    \section{What is Intel SGX}
    \begin{itemize}
        \item Protection of secrets (Personal ID, medical data, \dots) in isolated memory regions (enclaves)
        \item Enclaves are reserved from system memory and cannot be addressed. Additionally they are encrypted
        \item Applications can create enclaves and work inside them by using a "trusted function"
        \item Other attempts to access the enclave are denied by the processor
    \end{itemize}

    \section{Motivation}
    \begin{itemize}
        \item Process sensible data in a trusted environment on untrusted platforms:
            \begin{itemize}
                \item Protecting data even if the OS is malicious or against malware with kernel privileges
                \item Running operations on sensitive data in a cloud without having to trust the cloud provider
            \end{itemize}
        \item Faster than homomorphic encryption
        \item All main computations are performed on CPU, no need of communication with TPMs
    \end{itemize}

    \section{Idea of system design}
    Multiple mutually distrusting parties want to use a common service operating on sensible data (for example: Smart
    grid meters and service provider). Operations on data are performed by a trusted remote entity (TRE). The service
    provider has to trust the TRE that its calculations are correct, the smart meters have to trust the TRE that no personal
    data is compromised even if the entity running the TRE is malicious. Therefore all parties have to be able to ascertain 
    the code and operations running in the TRE (remote attestation).

    \section{Features of SGX}
    \begin{itemize}
        \item Isolation of areas of memory plus memory encryption
        \item Sealing of data, only allowing decryption by a specific enclave. This enables saving enclaves securely in 
              non volatile memory
        \item remote and local attestation of enclaves
    \end{itemize}

    \section{Principle of SGX}
    Enclaves are areas of virtual memory of an application process which are mapped to physically protected pages of real
    memory (encrypted in hardware). The mapping is done by the os but verified by the CPU which does not trust the OS in 
    general. Since the OS can alter the enclave binary before it is loaded, SGX uses remote attestation to ensure that only 
    correct code is run.

    \section{Security issues}
    SGX does not protect against side channel attacks: If the control flow or memory access pattern of the enclave
    depends on the secret data, sensible information may potentially be leaked. Further more, malicious code inside an 
    enclave is invisible to the OS and cannot be analyzed due to the SGX isolation, making it theoretically possible for 
    malware to hide itself in an enclave.

    Additionally, all enclaves lie on the same physical enclave page cache, thus enabling DRAM-bases side channel attacks. 

    \section{OpenSGX}
    Is a software emulation of Intel SGX supporting most instructions provided by the latter while also providing a 
    toolchain for monitoring performance and debugging support. An important aspect is that OpenSGX does not provide the
    same security guarantees as Intel SGX.

\end{document}