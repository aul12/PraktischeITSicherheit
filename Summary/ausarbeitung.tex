%pdflatex
%%%%%%%%%%%%%%%%%%%%%%%%%%%%%%%%%%%%%%%%%%%%%%%%%%%%%%%%%%%%%%%%%%%%%%%%%%%%%
%
% Vorlage für Seminararbeiten im Institut für Verteilte Systeme
% 
% HINWEISE
% 
%  1. Bei Nutzung für Seminarausarbeitungen darf insbesondere die Schriftart
%     und -größe nicht angepasst werden.
%  2. Die Vorlage unterstützt deutsche und englische Ausarbeitungen durch
%     Anpassung der Klassenoptionen.
%  3. Folgende Angaben sollen angepasst werden:
%     - Titel der Arbeit
%     - Name und E-Mail-Adresse der Autorin/des Autors
%     - Titel des Seminars
%     - Semester
%  4. Die Vorlage sieht eine Lizensierung unter CC-BY-SA vor, die jedoch
%     nicht verpflichtend ist. Falls nicht gewünscht, bitte Copyright auf
%     "retain" setzen.
%     Die gewählte Lizenz (CC-BY-SA) ist kompatibel mit einer möglichen
%     Veröffentlichung auf dem Volltextserver der Uni Ulm
%     (http://vts.uni-ulm.de).
%
%%%%%%%%%%%%%%%%%%%%%%%%%%%%%%%%%%%%%%%%%%%%%%%%%%%%%%%%%%%%%%%%%%%%%%%%%%%%%
%
% Template for seminar papers in the Institute of Distributed Systems
% 
% NOTES
% 
%  1. When used for seminar papers, the font type and size must not be changed.
%  2. The template supports German and English versions by modifying the class
%     options.
%  3. The following information should be adjusted:
%     - Title of the paper
%     - Name and mail address of the author
%     - Seminar title
%     - Semester
%  4. The template forsees licensing under CC-BY-SA, which however is not
%     mandatory. If not desired, please set the copyright to "retain".
%     The selected license (CC-BY-SA) is compatible with a possible
%     publication on the full text server of the Ulm University
%     (http://vts.uni-ulm.de).
%
%%%%%%%%%%%%%%%%%%%%%%%%%%%%%%%%%%%%%%%%%%%%%%%%%%%%%%%%%%%%%%%%%%%%%%%%%%%%%

% Based on the 2017 ACM Master Article Template
% \documentclass[vspaper,language=german]{acmart} % Deutsche Ausarbeitung
\documentclass[vspaper,language=english]{acmart} % Englische Ausarbeitung
\settopmatter{printfolios=true} % Adding Page numbers

% \setcopyright{retain}
% \setcopyright{cc-by}
\setcopyright{cc-by-sa}

%Seminar Akronym = KTT, PRIV, PASF, ATVS oder RTDS
% \acmConference[Seminar Akr.]{Titel des Seminars}{WS/SS XXXX}{Institut für Verteilte Systeme, Universität Ulm}
\acmConference[PSEC]{Praktische IT-Sicherheit}{SS 2020}{Institute of Distributed Systems, Ulm University}

\usepackage{import}
\usepackage{cleveref}
\usepackage[printonlyused]{acronym}

\newcommand{\incfig}[2][\columnwidth]{%
    \def\svgwidth{#1}
    \import{Images/}{#2.eps_tex}
}

\begin{document}

\title{Intel SGX Applicability and Security}

\author{Tim Luchterhand, Björn Petersen, Paul Nykiel} 
\email{tim.luchterhand@uni-ulm.de, bjoern.petersen@uni-ulm.de, paul.nykiel@uni-ulm.de}

\begin{abstract}
    Intel’s Software Guard Extensions (SGX) is a set of extensions to the Intel architecture that aims to provide integrity and confidentiality guarantees to security-sensitive
computation performed on a computer where all the privileged software (kernel, hypervisor, etc) is potentially malicious \cite{Costan2016IntelSE}. It includes new instructions
that allow a developer to place sensitive information inside isolated computation environments (so called Enclaves) where it is protected against unauthorized access even by
entities with a higher privilege level. In this report the basic principles of Intel SGX are explained. For a very detailed explanation please refer to \cite{Costan2016IntelSE}.
\end{abstract}
\hbadness=10000
\maketitle
\hbadness=1000

\section{Introduction} \label{sec:introduction}
Nowadays, outsourcing computation to remote entities is common practice, whether it be as part of an online service or 
in general as part of multi-party applications \cite{UseOfIntelSGX}. 
Alongside the benefits of increased flexibility and scalability this also introduces the challenge of performing secure 
computations in an untrusted and potentially malicious environment. 
This environment includes other software running on the same host but also the operating system and hypervisor running
on the machine.
It is difficult to guarantee the security of all data purely using software, especially when considering a potentially 
malicious, but more privileged, operating system or hypervisor.

Intel SGX provides a hardware based solution to this problem: new features and instructions are added to the processor
in order to store data in special memory which can not be accessed by unauthorized parties, even if they are more
privileged. Additionally Intel SGX provides the necessary instructions for attestation, so that it is possible to
verify the integrity of the program handling the confidential data.

This report first explains how Intel SGX works by explaining the mechanisms implemented by Intel, such as a
dedicated memory layout and new instructions. The next part explains local and remote attestation, two ways to
verify the integrity of an enclave. The last part describes sealing which can be used to store an enclave persistently.

\section{Intel SGX}
Intel SGX achieves its security guarantees by distinguishing between \textit{trusted} and \textit{untrusted} code \cite{IntelWebBasedTraining}. During development,
the programmer has to specify which part of the software belongs to which category. The former are functions that manage and process security critical data e.g. passwords,
encryption key etc. (the applications secrets) and in turn should be protected against unauthorized access. The latter are functions that perform standard tasks e.g. the
user interface and therefore do not directly interact with the applications secrets. To ensure the integrity and confidentiality of said secrets, trusted functions are 
placed in isolated execution environments which are referred to as \textit{enclaves} \cite{UseOfIntelSGX}. An application using Intel SGX consists of untrusted code and
one or more enclaves. Enclave code is compiled separately and is dynamically loaded at runtime by the host process, similarly to dynamic libraries \cite{Costan2016IntelSE}.
At runtime, the host process can instantiate an enclave using special processor instructions which ensure that enclaves are loaded correctly be the untrusted operating system
(OS). Once instantiated, the host process can enter enclaves -- again using special instructions -- and access the contained secrets and code. When done, the enclave is exited,
the host processes context is restored and the program execution is continued. Only when inside an enclave, the corresponding application secretes can be accessed and the
Intel SGX implementation prevents even high level privilege entities from illegally accessing an enclave's content.
\begin{figure}[h!]
    \centering
    \incfig[8cm]{AppOverview}
    \caption{Concept of Intel SGX. Graphic inspired by \cite{IntelSGXExplanation}.}
    \label{fig:enclaveCall}
\end{figure}

\subsection{Memory Layout of Intel SGX}
The SGX implementation stores sensitive data (enclave content, metadata) in an encrypted part of the system memory which is reserved for the processor only (Processor
Reserved Memory -- PRM). Direct access to the PRM is denied by the CPU, en- and decryption is preformed in hardware. The encryption keys are generated at boot time and
never leave the CPU \cite{IntelSGXExplanation}. The PRM itself holds enclave data (Enclave Page Cache -- EPC) and metadata (e.g. Enclave Page Cache Map -- EPCM) used 
to manage enclave creation, destruction and access.

\subsubsection{Enclave Page Cache}
The content of Intel SGX enclaves is stored in one or more Enclave Pages (EP) which reside inside the Enclave Page Cache (EPC) which itself is part of the EPC. An EP
can only be owned by exactly one enclave. It follows that different enclaves cannot communicate over shared EPs, however they can used standard untrusted memory to
do so \cite{Costan2016IntelSE}. Access between running enclaves and their corresponding EPs is managed by the Enclave Page Cache Map (EPCM). The EPCM also records
the corresponding virtual addresses for each EP. This information is used to ensure that page and address mapping, which is still performed by the untrusted OS, 
is according to the SGX specification (see \cref{sec:AddressTranslation}). 

\subsubsection{Enclave Control Structure}
Additional metadata of an enclave is stored in a special EP containing the SGX Enclave Control Structure (SECS). It contains among other a unique identifier, the
state of the corresponding enclave (un- / initialized), the enclave's size and a cryptographic hash over the enclave content called the enclave measurement. The latter
plays an important role during the creation of an enclave and is also used to derive cryptographic keys used for attestation and / or sealing. It is noteworthy that even
though the SECS is part of an enclave, it can only be accessed by the Intel SGX implementation and not from trusted functions. Allowing access even by trusted code would
pose the risk of corrupting the SGX memory layout or corruption of cryptographic secrets \cite{Costan2016IntelSE}.

\subsubsection{Extended Address Translation and Page Mapping}
\label{sec:AddressTranslation}
Intel SGX adds on top of the standard address mapping between virtual and physical addresses performed by the OS \cite{Costan2016IntelSE}. Since the PRM is still part of the
system memory address translation and page mapping can be delegated to the OS, minimizing the the amount of changes required to add SGX support to existing system software
\cite{IntelSGXSSLab}. However, given that the OS itself is not trusted, additional checks have to be performed by the CPU to ensure the integrity and confidentiality of PRM
data. In particular the SGX implementation uses metadata stored within the PRM (such as the EPCM) to ensure that:
\begin{enumerate}
    \item Virtual addresses pointing to enclave code or data can only be mapped to EPs. This ensures that enclave content is always protected inside the PRM and not mapped to
          standard unencrypted memory.
    \item EPs can only be mapped to one specific virtual address. This implies a one-to-one mapping between EPs and virtual addresses and guarantees that only virtual addresses
          located within the address space of an enclave can access the enclave content.
    \item EPs can only be mapped to one specific virtual address and therefore no more than one enclave has access to a specific EP.
\end{enumerate}
Further more, standard address translation is extended by subsequent checks: If code running within an untrusted context requests access to a memory page inside the EPC,
the CPU loads a so called abort page signaling the application that the requested page does not exist \cite{Costan2016IntelSE}. If however an EP is requested from within a
running enclave, the CPU checks whether the enclave owns the requested page using the EPCM. Only if the check succeeds the access is allowed.

\begin{figure}[h!]
    \centering
    \incfig[8cm]{PageCheck}
    \caption{Extended Checks on Address Translation (see \cite{IntelSGXExplanation})}
    \label{fig:pageCheck}
\end{figure}
\subsection{Enclave Signature}
\label{sec:EnclaveSignature}
"Enclaves include a self-signed certificate from the enclave author, also known as the \textit{enclave signature}. It contains information that allows the Intel SGX architecture 
to detect whether any portion of the enclave file has been tampered with. This allows an enclave to prove that it has been loaded in EPC correctly and it can be trusted" (Quoted
from \cite{IntelEnclaveSignature}). The signature contains among other information:
\begin{itemize}
    \item The \textit{enclave measurement} which is an SHA-2 hash \cite{Costan2016IntelSE} which identifies data and code of an enclave. During enclave creation the enclave
          measurement is updated using the EEXTEND instruction. When the enclave is ready to be initialized, the hash is compared against the expected measurement specified
          in the enclave's signature. A mismatch will prevent the enclave from executing \cite{IntelEnclaveSignature}.
    \item The author's public key and signature
    \item Different version numbers i.e. a product ID which can be used during attestation and a security version number which reflects the security property of the enclave.
\end{itemize}
The enclave signature is not only used to verify that the enclave has been loaded correctly (by comparing the enclave measurement) but also during attestation.
\subsection{Special Processor Features}
Intel processors supporting Intel SGX have an additional processor mode as well as new instructions which are used to create, tear down and communicate with enclaves.

\subsubsection{Enclave Mode}
\label{sec:EnclaveMode}
When executing enclave code the processor is said to be in \textit{enclave mode} \cite{Costan2016IntelSE}. In order to access memory pages within the EPC, the processor has to
be in enclave mode. While in enclave mode, untrusted memory can still be accessed. Untrusted application code on the other hand runs in \textit{untrusted mode} and can only
access untrusted memory. Using special instructions the processor modes can be switched. This behaves similarly to a system call used to switch from user mode to kernel mode in 
order to execute code that requires kernel level privileges. However, code running in enclave mode still (and always) runs at user level privileges to ensure that the standard 
security mechanisms of the OS can be applied \cite{Costan2016IntelSE}.

\subsubsection{Special Instructions}
\label{sec:Instructions}
Intel SGX introduces a set of new instructions which have to be supported by the processor. The following is an excerpt of the most important ones \cite{OverviewOfIntelSGX}:
\begin{enumerate}
    \item EENTER, EEXIT are used to enter or exit enclave mode respectively. Both can only be called from user mode and do not perform a privilege level switch as explained
          in \cref{sec:EnclaveMode}.
    \item ECREATE, EREMOVE, EINIT, EADD, EEXTEND are used during the creation or destruction of enclaves. The require kernel level privileges as they involve (de-)allocation
          of memory pages to or from the EPC.
    \item EREPORT and EGETKEY are used to derive cryptographic signatures used during the attestation process. They require the processor to be in enclave mode and therefore
          require user level privileges.
\end{enumerate}
\subsection{Enclave Life Cycle}
The following section gives a brief overview about the different states of an enclave. Before an enclave can be used, it needs to be explicitly created (loaded) and initialized.
Once completely initialized, using EENTER and EEXIT, the enclave can be called from untrusted code. Unused enclaves can also be destructed be using EREMOVE. The live cycle of an
enclave is visualized in \cref{fig:enclaveLifeCycle}.

\begin{figure}[h]
    \centering
    \incfig[8cm]{EnclaveLifeCycle}
    \caption{Different states of an Intel SGX enclave \cite{Costan2016IntelSE}}
    \label{fig:enclaveLifeCycle}
\end{figure}

\subsubsection{Enclave Creation}
To instantiate an Intel SGX enclave several steps have to be taken. The creation process of an enclave using instructions introduced in \cref{sec:Instructions} involves allocation
of EP pages and thus cannot be initiated from user mode directly. Instead it is delegated to the (untrusted) OS. To ensure that the OS loads creates enclaves correctly and
especially does not temper with an enclave's content, the SGX implementation observes this process using an enclave's signature. The creation process works as follows
\cite{Costan2016IntelSE}:
\begin{enumerate}
    \item Using the ECREATE instruction a free EP is used to store the SECS of the to be created enclave. The SECS is marked as \textit{uninitialized}.
    \item EADD is used to load the initial code and data into EPs.
    \item Each page added with EADD is added to the enclave measurement using EEXTEND.
    \item EINIT is used to finalize the enclave creation process. This includes comparing the enclave measurement against the expected measurement to make sure that the enclave's
          content has been loaded correctly. The SECS is marked as \textit{initialized} and the enclave is now ready to use.
\end{enumerate}

\subsubsection{Calling an Enclave}
An initialized enclave can be entered using the EENTER instruction. As stated previously this can only be done from user mode. Similarly to a syscall the context of the current
process is saved (registers, instruction pointer, \dots). The processor then switches to enclave mode and executes the desired code inside the enclave. When done the host processes
execution can be resumed by using EEXIT. This causes the processor to switch to normal mode and restore the hosts processes context. Apart from a synchronous entry and exit there
exists also the possibility to exit an enclave asynchronously. This will happen when a hardware fault or interrupt occurs during execution of an enclave's code. The SGX
implementation will then save the enclave code's execution context, restore the the state saved by EENTER and set up the processors registers so that the system software’s hardware 
exception handler will return to an asynchronous exit handler in the enclave’s host process \cite{Costan2016IntelSE}. The enclave computation is resumed by using the ERESUME
instruction from the exit handler. Further details of asynchronous enclave exits and recovering from said exits is described in \cite{Costan2016IntelSE}.

\subsubsection{Enclave Destruction}
The EREMOVE instruction is used by the OS to destroy unused enclaves by deallocating their respective EPs. An enclave is completely destroyed when the EP holding its SECS is freed
\cite{Costan2016IntelSE}. The last step can only be performed after all other EPs have already been deallocated.
\section{Software Attestation}
Once created, third parties cannot directly access an enclave. To still be able to identify and verify the software running in an enclave, Intel SGX provides an attestation process.
Enclaves basically are identified by their signature which among other contains the enclave's measurement which is computed over its data and code. A third party requesting
authentication can then verify the measurement reported by the trusted hardware (which is the processor) and compare it against some expected measurement \cite{Costan2016IntelSE}.
Intel SGX supports two forms of attestation: local and remote attestation.

\subsection{Local Attestation}
The local attestation process takes place between two enclaves on the same platform, where the attestation of the \textit{target} is requested by the \textit{challenger}. The 
challenger asks the target to generate a report which can be used to uniquely identify the enclave and its contents. To do so, the target calls the EREPORT instruction. The report 
generated this way contains data from the target enclave's SECS, specifically (excerpt):
\begin{itemize}
    \item the target enclave's measurement
    \item the author's ID
    \item different version numbers
\end{itemize}
The report also includes a Message Authentication Code (MAC) computed over all fields included in the report. To compute the MAC, two additional secrets have to be supplied.
The first being a cryptographic secret only known by the processor, the second being the measurement of the challenger enclave. After the report is generated, the challenger can
verify its authenticity by using the EGETKEY instruction. It will recompute the MAC and compare it against the MAC field in the report. When both values match, the challenger can
be sure that the report is authentic for the following reasons: Firstly, it must have come from the target because it contains the targets measurement, and only when EREPORT is 
called from the target, the instruction has access said measurement. Secondly, it was definitely generated by the trusted SGX implementation because this is the only entity that 
has access to the processor's secret. And thirdly, only the challenger can verify the report because it is the only entity that can supply the required measurement to recompute the 
MAC of the report. This process works due to the fact that the target has to specify the enclave to which the report is destined (in this case the challenger) when calling EREPORT.
Only through EREPORT the target implicitly has access to the challenger's measurement which is used to compute the MAC the first time.

\section{Sealing}
Once an enclave is destroyed, all contained secrets are gone as well. Intel \ac{sgx} thus enables enclaves to seal data for persistent storage in a secure fashion. In the abstract a
secret unique to the platform (embedded in the processor) and an enclave's signature are used to generate the cryptographic material which is then used to derive encryption keys
for encrypting the data to be stored. Intel distinguishes between two main different sealing policies (\cite{EnclaveWritersGuide}, \cite{IntelSealing}):
\begin{enumerate}
    \item Sealing to one particular enclave: Using an enclave's measurement for the key derivation, the sealed data can only be recovered by another enclave with an identical
          measurement and thus identical structure. Even a different build or version of the same enclave will result in a different enclave measurement and prevent the enclave
          from unsealing the data again.
    \item Sealing to enclaves with the same signing identity: When requiring less strict access to the sealed data, instead of the enclave's measurement, other fields in its
          signature are used to derive the sealing key. Included are the author's identity, the product ID and security version numbers. This enables enclaves by the same
          author and the same product to share sealed secrets. It also means that when updating enclave code, the sealed data does not have to be unsealed by the old version
          and resealed by the new version, providing greater flexibility to the developer.
\end{enumerate}
In both cases, the sealing key is obtained by the EGETKEY instruction which has access to the enclave's \ac{secs} and the secret stored inside the processor. The key derivation material
is hereby fed to an AES-CMAC which provides a 128-bit symmetric key \cite{Costan2016IntelSE}. Since the key derivation process includes a secret embedded in the processor, it is
impossible to reproduce the same key on a different platform even if the key derivation material is known.


\appendix
\hbadness=10000
\bibliographystyle{ACM-Reference-Format}
\nocite{*}
\bibliography{references}
\section*{Acronyms}
\footnotesize{\begin{acronym}[EPCM]
    \acro{epcm}[EPCM]{Enclave Page Cache Map}
    \acro{epc}[EPC]{Enclave Page Cache}
    \acro{ep}[EP]{Enclave Page}
    \acro{os}[OS]{operating system}
    \acro{prm}[PRM]{Processor Reserved Memory}
    \acro{secs}[SECS]{SGX Enclave Control Structure}
    \acro{sgx}[SGX]{Software Guard Extension}
\end{acronym}
}
\hbadness=1000
\end{document}
