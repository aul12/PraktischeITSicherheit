\section{Introduction} \label{sec:introduction}
Nowadays, outsourcing computation to remote entities is common practice, whether it be as part of an online service or 
in general as part of multi-party applications \cite{UseOfIntelSGX}. 
Alongside the benefits of increased flexibility and scalability this also introduces the challenge of performing secure 
computations in an untrusted and potentially malicious environment. 
This environment includes other software running on the same host but also the operating system and hypervisor running
on the machine.
It is difficult to guarantee the security of all data purely using software, especially when considering a potentially 
malicious, but more privileged, operating system or hypervisor.

Intel SGX provides a hardware based solution to this problem: new features and instructions are added to the processor
in order to store data in special memory which can not be accessed by unauthorized parties, even if they are more
privileged. Additionally Intel SGX provides the necessary instructions for attestation, so that it is possible to
verify the integrity of the program handling the confidential data.

This report first explains how Intel SGX works by explaining the mechanisms implemented by Intel, such as a
dedicated memory layout and new instructions. The next part explains local and remote attestation, two ways to
verify the integrity of an enclave. The last part describes sealing which can be used to store an enclave persistently.
