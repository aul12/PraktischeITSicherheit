\section{Intel SGX}
Intel SGX achieves its security guarantees by distinguishing between \textit{trusted} and \textit{untrusted} code \cite{IntelWebBasedTraining}. During development,
the programmer has to specify which part of the software belongs to which category. The former are functions that manage and process security critical data e.g. passwords,
encryption key etc. (the applications secrets) and in turn should be protected against unauthorized access. The latter are functions that perform standard tasks e.g. the
user interface and therefore do not directly interact with the applications secrets. To ensure the integrity and confidentiality of said secrets, trusted functions are 
placed in isolated execution environments which are referred to as \textit{enclaves} \cite{UseOfIntelSGX}. An application using Intel SGX consists of untrusted code and
one or more enclaves. Enclave code is compiled separately and is dynamically loaded at runtime by the host process, similarly to dynamic libraries \cite{Costan2016IntelSE}.
At runtime, the host process can instantiate an enclave using special processor instructions which ensure that enclaves are loaded correctly be the untrusted operating system.
Once instantiated, the host process can enter enclaves -- again using special instructions -- and access the contained secrets and code. When done, the enclave is exited,
the host processes context is restored and the program execution is continued. Only when inside an enclave, the corresponding application secretes can be accessed and the
Intel SGX implementation prevents even high level privilege entities from illegally accessing an enclave's content.
\begin{figure}[h!]
    \centering
    \incfig[9.5cm]{AppOverview8}
    \caption{@TODO fix scale}
    \label{fig:enclaveCall}
\end{figure}

\subsection{Memory Layout of Intel SGX}
The SGX implementation stores sensitive data (enclave content, metadata) in an encrypted part of the system memory which is reserved for the processor only (Processor
Reserved Memory -- PRM). Direct access to the PRM is denied by the CPU, en- and decryption is preformed in hardware. The encryption keys are generated at boot time and
never leave the CPU \cite{IntelSGXExplanation}. The PRM itself holds enclave data (Enclave Page Cache -- EPC) and metadata (e.g. Enclave Page Cache Map -- EPCM) used 
to manage enclave creation, destruction and access.