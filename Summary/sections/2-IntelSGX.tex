\section{Intel SGX}
Intel \ac{sgx} achieves its security guarantees by distinguishing between \textit{trusted} and \textit{untrusted} code \cite{IntelWebBasedTraining}. During development,
the programmer has to specify which part of the software belongs to which category. The former are functions that manage and process security-critical data e.g. passwords,
encryption keys etc. (the applications secrets) and in turn should be protected against unauthorized access. The latter are functions that perform security uncritical 
tasks e.g. the user interface and therefore do not directly interact with the applications secrets. To ensure the integrity and confidentiality of said secrets, trusted 
functions are placed in isolated execution environments which are referred to as \textit{enclaves} \cite{UseOfIntelSGX}. An application using Intel \ac{sgx} consists of untrusted 
code and one or more enclaves. Enclave code is compiled separately and is dynamically loaded at runtime by the host process, similarly to dynamic libraries \cite{Costan2016IntelSE}.
At runtime, the host process can instantiate an enclave using special processor instructions which ensure that enclaves are loaded correctly by the untrusted \ac{os}. 
Once instantiated, the host process can enter enclaves -- again using special instructions -- and access the contained secrets and code. When done, the enclave is exited,
the host processes context is restored and the program execution is continued. Only when inside an enclave, the corresponding application secretes can be accessed and the
Intel \ac{sgx} implementation prevents even high level privilege entities from illegally accessing an enclave's content.
\begin{figure}[h!]
    \centering
    \incfig[8cm]{AppOverview}
    \caption{Concept of Intel \ac{sgx}. Graphic inspired by \cite{IntelSGXExplanation}.}
    \label{fig:enclaveCall}
\end{figure}

\subsection{Memory Layout of Intel SGX}
The SGX implementation stores sensitive data (enclave content, metadata) in an encrypted part of the system memory which is reserved for the processor only (Processor
Reserved Memory -- PRM). Direct access to the PRM is denied by the CPU, en- and decryption is preformed in hardware. The encryption keys are generated at boot time and
never leave the CPU \cite{IntelSGXExplanation}. The PRM itself holds enclave data (Enclave Page Cache -- EPC) and metadata (e.g. Enclave Page Cache Map -- EPCM) used 
to manage enclave creation, destruction and access.

\subsubsection{Enclave Page Cache}
The content of Intel SGX enclaves is stored in one or more Enclave Pages (EP) which reside inside the Enclave Page Cache (EPC) which itself is part of the EPC. An EP
can only be owned by exactly one enclave. It follows that different enclaves cannot communicate over shared EPs, however they can used standard untrusted memory to
do so \cite{Costan2016IntelSE}. Access between running enclaves and their corresponding EPs is managed by the Enclave Page Cache Map (EPCM). The EPCM also records
the corresponding virtual addresses for each EP. This information is used to ensure that page and address mapping, which is still performed by the untrusted OS, 
is according to the SGX specification. In particular the SGX implementation ensures that:
\begin{enumerate}
    \item Virtual addresses pointing to enclave code or data can only be mapped to EPs. This ensures that enclave content is always protected inside the PRM and not stored in
          standard unencrypted memory.
    \item EPs can only be mapped to one specific virtual address. This implies a one-to-one mapping between EPs and virtual addresses and guarantees that only virtual addresses
          located within the address space of an enclave can access the enclave content.
    \item EPs can only be mapped to one specific virtual address and therefore no more than one enclave has access to a specific EP.
\end{enumerate}

\subsubsection{Enclave Control Structure}
Additional metadata of an enclave is stored in a special EP containing the SGX Enclave Control Structure (SECS). It contains among other a unique identifier, the
state of the corresponding enclave (un- / initialized), the enclave's size and a cryptographic hash over the enclave content called the enclave measurement. The latter
plays an important role during the creation of an enclave and is also used to derive cryptographic keys used for attestation and / or sealing. It is noteworthy that even
though the SECS is part of an enclave, it can only be accessed by the Intel SGX implementation and not from trusted functions. Allowing access even by trusted code would
pose the risk of corrupting the SGX memory layout or corruption of cryptographic secrets.
\subsection{Enclave Signature}
\label{sec:EnclaveSignature}
"Enclaves include a self-signed certificate from the enclave author, also known as the \textit{enclave signature}. It contains information that allows the Intel SGX architecture 
to detect whether any portion of the enclave file has been tampered with. This allows an enclave to prove that it has been loaded in EPC correctly and it can be trusted" (Quoted
from \cite{IntelEnclaveSignature}). The signature contains among other information:
\begin{itemize}
    \item The \textit{enclave measurement} which is an SHA-2 hash \cite{Costan2016IntelSE} which identifies data and code of an enclave. During enclave creation the enclave
          measurement is updated using the EEXTEND instruction. When the enclave is ready to be initialized, the hash is compared against the expected measurement specified
          in the enclave's signature. A mismatch will prevent the enclave from executing \cite{IntelEnclaveSignature}.
    \item The author's public key and signature
    \item Different version numbers i.e. a product ID which can be used during attestation and a security version number which reflects the security property of the enclave.
\end{itemize}
The enclave signature is not only used to verify that the enclave has been loaded correctly (by comparing the enclave measurement) but also during attestation.
\subsection{Special Processor Features}
Intel processors supporting Intel SGX have an additional processor mode as well as new instructions which are used to create, tear down and communicate with enclaves.

\subsubsection{Enclave Mode}
\label{sec:EnclaveMode}
When executing enclave code the processor is said to be in \textit{enclave mode} \cite{Costan2016IntelSE}. In order to access memory pages within the EPC, the processor has to
be in enclave mode. While in enclave mode, untrusted memory can still be accessed. Untrusted application code on the other hand runs in \textit{untrusted mode} and can only
access untrusted memory. Using special instructions the processor modes can be switched. This behaves similarly to syscall when switching from user mode to kernel mode in order
to execute code that requires kernel level privileges. However, code running in enclave mode still (and always) runs at user level privileges to ensure that the standard security
mechanisms of the OS can be applied \cite{Costan2016IntelSE}.

\subsubsection{Special Instructions}
Intel SGX introduces a set of new instructions which have to be supported by the processor. The following is an excerpt of the most important ones:
\begin{enumerate}
    \item EENTER, EEXIT are used to enter or exit enclave mode respectively. Both can only be called from user mode and do not perform a privilege level switch as explained
          in \cref{sec:EnclaveMode}.
    \item ECREATE, EREMOVE, EINIT, EADD, EEXTEND are used during the creation or destruction of enclaves. The require kernel level privileges as they involve (de-)allocation
          of memory pages to or from the EPC.
    \item EREPORT and EGETKEY are used to derive cryptographic signatures used during the attestation process. They require the processor to be in enclave mode and therefore
          require user level privileges.
\end{enumerate}
\subsection{Enclave Life Cycle}
The following section gives a brief overview about the different states of an enclave. Before an enclave can be used, it needs to be explicitly created (loaded) and initialized.
Once completely initialized, using EENTER and EEXIT, the enclave code can be called from untrusted code. Unused enclaves can also be destructed be using EREMOVE. The live cycle 
of an enclave is visualized in \cref{fig:enclaveLifeCycle}.

\begin{figure}[h]
    \centering
    \incfig[8cm]{EnclaveLifeCycle}
    \caption{Different states of an Intel \ac{sgx} enclave \cite{Costan2016IntelSE}}
    \label{fig:enclaveLifeCycle}
\end{figure}

\subsubsection{Enclave Creation}
To instantiate an Intel \ac{sgx} enclave several steps have to be taken. The creation process of an enclave using instructions introduced in \cref{sec:Instructions} involves allocation
of \ac{ep} pages and thus cannot be initiated from user mode directly. Instead it is delegated to the (untrusted) \ac{os}. To ensure that the \ac{os} loads and creates enclaves correctly and
especially does not temper with an enclave's content, the \ac{sgx} implementation observes this process using an enclave's signature. The creation process works as follows
\cite{Costan2016IntelSE}:
\begin{enumerate}
    \item Using the ECREATE instruction a free \ac{ep} is used to store the \ac{secs} of the enclave that is to be created. The \ac{secs} is marked as \textit{uninitialized}.
    \item EADD is used to load the initial code and data into \acp{ep}.
    \item Each page that was added with EADD, is added to the enclave measurement using EEXTEND.
    \item EINIT is used to finalize the enclave creation process. This includes comparing the enclave measurement against the expected measurement to make sure that the enclave's
          content has been loaded correctly. The \ac{secs} is marked as \textit{initialized} and the enclave is now ready to use.
\end{enumerate}

\subsubsection{Calling an Enclave}
An initialized enclave can be entered using the EENTER instruction. As stated previously, this can only be done from user mode. Similarly to a system call the context of the current
process is saved (registers, instruction pointer, \dots). The processor then switches to enclave mode and executes the desired code inside the enclave. When done the host processes
execution can be resumed by using EEXIT. This causes the processor to switch to normal mode and restore the host processes context. Apart from a synchronous entry and exit there
exists also the possibility to exit an enclave asynchronously. This will happen when a hardware fault or interrupt occurs during execution of an enclave's code. The \ac{sgx}
implementation will then save the enclave code's execution context, restore the state saved by EENTER and set up the processors registers so that the system software’s hardware 
exception handler will return to an asynchronous exit handler in the enclave’s host process \cite{Costan2016IntelSE}. The enclave computation is resumed by using the ERESUME
instruction from the exit handler. Further details of asynchronous enclave exits and recovering from said exits is described in \cite{Costan2016IntelSE}.

\subsubsection{Enclave Destruction}
The EREMOVE instruction is used by the \ac{os} to destroy unused enclaves by deallocating their respective \acp{ep}. An enclave is completely destroyed when the \ac{ep} holding its \ac{secs} is freed
\cite{Costan2016IntelSE}. The last step can only be performed after all other \acp{ep} have already been deallocated.

