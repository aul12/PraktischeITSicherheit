\subsection{Memory Layout of Intel \ac{sgx}}
Intel \ac{sgx} stores sensitive data (enclave content, metadata) in an encrypted part of the system memory which is reserved for and can only be accessed by the \ac{sgx} 
implementation. This part of the system memory is referred to as \ac{prm}. Direct access to the \ac{prm} is denied by the CPU, en- and decryption 
is preformed in hardware. The encryption keys are generated at boot time and never leave the CPU \cite{IntelSGXExplanation}. The \ac{prm} itself holds enclave data (\ac{epc}) and metadata (e.g. \ac{epcm}) used to manage enclave creation, destruction and access.

\subsubsection{Enclave Page Cache}
The content of Intel \ac{sgx} enclaves is stored in one or more \ac{ep} which reside inside the \ac{epc} which itself is part of the \ac{epc}. An \ac{ep}
can only be owned by exactly one enclave. It follows that different enclaves cannot communicate over shared \acp{ep}, however they can used standard untrusted memory to
do so \cite{Costan2016IntelSE}. Access between running enclaves and their corresponding \acp{ep} is managed by the \ac{epcm}. The \ac{epcm} also records
the corresponding virtual addresses for each \ac{ep}. This information is used to ensure that page and address mapping, which is still performed by the untrusted \ac{os}, 
are according to the \ac{sgx} specification (see \cref{sec:AddressTranslation}). 

\subsubsection{Enclave Control Structure}
Additional metadata of an enclave is stored in a special \ac{ep} containing the \ac{secs}. It contains for example a unique identifier, the
state of the corresponding enclave (un- / initialized), the enclave's size and a cryptographic hash over the enclave content called the \textit{enclave measurement} (see
also \cref{sec:EnclaveSignature}). The latter plays an important role during the creation of an enclave and is also used to derive cryptographic keys used for attestation 
and / or sealing. It is noteworthy, that even though the \ac{secs} is part of an enclave, it can only be accessed by the Intel \ac{sgx} implementation and not from trusted functions. 
Allowing access even by trusted code would pose the risk of corrupting the \ac{sgx} memory layout or exposure of cryptographic secrets \cite{Costan2016IntelSE}.

\subsubsection{Extended Address Translation and Page Mapping}
\label{sec:AddressTranslation}
Intel \ac{sgx} adds  on top of the standard address mapping between virtual and physical addresses performed by the \ac{os} \cite{Costan2016IntelSE}. Since the \ac{prm} is still part of the 
system memory, address translation and page mapping can be delegated to the \ac{os}, minimizing the the amount of changes required to add \ac{sgx} support to existing system software 
\cite{IntelSGXSSLab}. However, given that the \ac{os} itself is not trusted, additional checks have to be performed by the CPU to ensure the integrity and confidentiality of \ac{prm} 
data. In particular, the \ac{sgx} implementation uses metadata stored within the \ac{prm} (such as the \ac{epcm}) to ensure that:
\begin{enumerate}
    \item Virtual addresses pointing to enclave code or data can only be mapped to \acp{ep}. This ensures that enclave content is always protected inside the \ac{prm} and not mapped to
          standard unencrypted memory.
    \item \acp{ep} can only be mapped to one specific virtual address. This implies a one-to-one mapping between \acp{ep} and virtual addresses and guarantees that only virtual addresses
          located within the address space of an enclave can access the enclave content.
    \item \acp{ep} can only be mapped to one specific virtual address and therefore no more than one enclave has access to a specific \ac{ep}.
\end{enumerate}
Furthermore, standard address translation is extended by subsequent checks: if code running within an untrusted context requests access to a memory page inside the \ac{epc},
the CPU loads a so called abort page signaling the application that the requested page does not exist \cite{Costan2016IntelSE}. If however an \ac{ep} is requested from within a
running enclave, the CPU checks whether the enclave owns the requested page using the \ac{epcm}. Only if the check succeeds, the access is allowed.

\begin{figure}[h!]
    \centering
    \incfig[8cm]{PageCheck}
    \caption{Extended Checks on Address Translation (see \cite{IntelSGXExplanation})}
    \label{fig:pageCheck}
\end{figure}
