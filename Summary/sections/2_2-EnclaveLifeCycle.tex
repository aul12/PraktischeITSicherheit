\subsection{Enclave Life Cycle}
The following section gives a brief overview about the different states of an enclave. Before an enclave can be used, it needs to be explicitly created (loaded) and initialized.
Once completely initialized, using EENTER and EEXIT, the enclave can be called from untrusted code. Unused enclaves can also be destructed be using EREMOVE. The live cycle of an
enclave is visualized in \cref{fig:enclaveLifeCycle}.

\begin{figure}[h]
    \centering
    \incfig[8cm]{EnclaveLifeCycle}
    \caption{Different states of an Intel SGX enclave \cite{Costan2016IntelSE}}
    \label{fig:enclaveLifeCycle}
\end{figure}

\subsubsection{Enclave Creation}
To instantiate an Intel SGX enclave several steps have to be taken. The creation process of an enclave using instructions introduced in \cref{sec:Instructions} involves allocation
of EP pages and thus cannot be initiated from user mode directly. Instead it is delegated to the (untrusted) OS. To ensure that the OS loads creates enclaves correctly and
especially does not temper with an enclave's content, the SGX implementation observes this process using an enclave's signature.

\subsubsection{Calling an Enclave}

\subsubsection{Enclave Destruction}