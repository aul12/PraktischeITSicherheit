\subsection{Enclave Life Cycle}
The following section gives a brief overview about the different states of an enclave. Before an enclave can be used, it needs to be explicitly created (loaded) and initialized.
Once completely initialized, using EENTER and EEXIT, the enclave code can be called from untrusted code. Unused enclaves can also be destructed be using EREMOVE. The live cycle 
of an enclave is visualized in \cref{fig:enclaveLifeCycle}.

\begin{figure}[h]
    \centering
    \incfig[8cm]{EnclaveLifeCycle}
    \caption{Different states of an Intel \ac{sgx} enclave \cite{Costan2016IntelSE}}
    \label{fig:enclaveLifeCycle}
\end{figure}

\subsubsection{Enclave Creation}
To instantiate an Intel \ac{sgx} enclave several steps have to be taken. The creation process of an enclave using instructions introduced in \cref{sec:Instructions} involves allocation
of \ac{ep} pages and thus cannot be initiated from user mode directly. Instead it is delegated to the (untrusted) \ac{os}. To ensure that the \ac{os} loads and creates enclaves correctly and
especially does not temper with an enclave's content, the \ac{sgx} implementation observes this process using an enclave's signature. The creation process works as follows
\cite{Costan2016IntelSE}:
\begin{enumerate}
    \item Using the ECREATE instruction a free \ac{ep} is used to store the \ac{secs} of the enclave that is to be created. The \ac{secs} is marked as \textit{uninitialized}.
    \item EADD is used to load the initial code and data into \acp{ep}.
    \item Each page that was added with EADD, is added to the enclave measurement using EEXTEND.
    \item EINIT is used to finalize the enclave creation process. This includes comparing the enclave measurement against the expected measurement to make sure that the enclave's
          content has been loaded correctly. The \ac{secs} is marked as \textit{initialized} and the enclave is now ready to use.
\end{enumerate}

\subsubsection{Calling an Enclave}
An initialized enclave can be entered using the EENTER instruction. As stated previously, this can only be done from user mode. Similarly to a system call the context of the current
process is saved (registers, instruction pointer, \dots). The processor then switches to enclave mode and executes the desired code inside the enclave. When done the host processes
execution can be resumed by using EEXIT. This causes the processor to switch to normal mode and restore the host processes context. Apart from a synchronous entry and exit there
exists also the possibility to exit an enclave asynchronously. This will happen when a hardware fault or interrupt occurs during execution of an enclave's code. The \ac{sgx}
implementation will then save the enclave code's execution context, restore the state saved by EENTER and set up the processors registers so that the system software’s hardware 
exception handler will return to an asynchronous exit handler in the enclave’s host process \cite{Costan2016IntelSE}. The enclave computation is resumed by using the ERESUME
instruction from the exit handler. Further details of asynchronous enclave exits and recovering from said exits is described in \cite{Costan2016IntelSE}.

\subsubsection{Enclave Destruction}
The EREMOVE instruction is used by the \ac{os} to destroy unused enclaves by deallocating their respective \acp{ep}. An enclave is completely destroyed when the \ac{ep} holding its \ac{secs} is freed
\cite{Costan2016IntelSE}. The last step can only be performed after all other \acp{ep} have already been deallocated.
