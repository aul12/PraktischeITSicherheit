\subsection{Special Processor Features}
Intel processors supporting Intel SGX have an additional processor mode as well as new instructions which are used to create, tear down and communicate with enclaves.

\subsubsection{Enclave Mode}
\label{sec:EnclaveMode}
When executing enclave code the processor is said to be in \textit{enclave mode} \cite{Costan2016IntelSE}. In order to access memory pages within the EPC, the processor has to
be in enclave mode. While in enclave mode, untrusted memory can still be accessed. Untrusted application code on the other hand runs in \textit{untrusted mode} and can only
access untrusted memory. Using special instructions the processor modes can be switched. This behaves similarly to syscall when switching from user mode to kernel mode in order
to execute code that requires kernel level privileges. However, code running in enclave mode still (and always) runs at user level privileges to ensure that the standard security
mechanisms of the OS can be applied \cite{Costan2016IntelSE}.

\subsubsection{Special Instructions}
Intel SGX introduces a set of new instructions which have to be supported by the processor. The following is an excerpt of the most important ones:
\begin{enumerate}
    \item EENTER, EEXIT are used to enter or exit enclave mode respectively. Both can only be called from user mode and do not perform a privilege level switch as explained
          in \cref{sec:EnclaveMode}.
    \item ECREATE, EREMOVE, EINIT, EADD, EEXTEND are used during the creation or destruction of enclaves. The require kernel level privileges as they involve (de-)allocation
          of memory pages to or from the EPC.
    \item EREPORT and EGETKEY are used to derive cryptographic signatures used during the attestation process. They require the processor to be in enclave mode and therefore
          require user level privileges.
\end{enumerate}