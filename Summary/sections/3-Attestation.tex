\section{Software Attestation}
Once created, third parties cannot directly access an enclave. Intel SGX provides an attestation process, so that it can still identify and verify the software running in an enclave.
Enclaves basically are identified by their signature which among other contains the enclave's measurement which is computed over its data and code. A third party requesting
authentication can then verify the measurement reported by the trusted hardware (which is the processor) and compare it against some expected measurement \cite{Costan2016IntelSE}.
Intel SGX supports two forms of attestation: local and remote attestation. The following sections give a brief overview of both attestation schemes. The reader is encouraged to
further study the details presented in \cite{Costan2016IntelSE} and \cite{EnclaveWritersGuide}.

\subsection{Local Attestation}
The local attestation process takes place between two enclaves on the same platform, where the attestation of the \textit{target} is requested by the \textit{challenger}. The 
challenger asks the target to generate a report which can be used to uniquely identify the enclave and its contents. To do so, the target calls the EREPORT instruction. The report 
generated this way contains data from the target enclave's SECS, specifically (excerpt of \cite{EnclaveWritersGuide}):
\begin{itemize}
    \item the target enclave's measurement
    \item the author's ID
    \item different version numbers
\end{itemize}
The report also includes a Message Authentication Code (MAC) computed over all fields included in the report. To compute the MAC, two additional secrets have to be supplied.
The first being a cryptographic secret only known by the processor, the second being the measurement of the challenger enclave. After the report is generated, the challenger can
verify its authenticity by using the EGETKEY instruction. It will recompute the MAC and compare it against the MAC field in the report. When both values match, the challenger can
be sure that the report is authentic for the following reasons: Firstly, it must have come from the target because it contains the targets measurement, and only when EREPORT is 
called from the target, the instruction has access said measurement. Secondly, it was definitely generated by the trusted SGX implementation because this is the only entity that 
has access to the processor's secret. And thirdly, only the challenger can verify the report because it is the only entity that can supply the required measurement to recompute the 
MAC of the report. This process works due to the fact that the target has to specify the enclave to which the report is destined (in this case the challenger) when calling EREPORT.
Only through EREPORT the target implicitly has access to the challenger's measurement which is used to compute the MAC the first time.

\subsection{Remote Attestation}
Intel SGX also supports attestation between different systems. For this process, both the SGX enabled processor on the target system and Intel have to share a secret. This means
that Intel basically takes the role of a certificate authority. The remote attestation itself mostly relies on the local attestation process described above using an additional 
\textit{quoting enclave} which takes the role of the challenger on the target system. The quoting enclave verifies the report received from the target enclave and signs it using 
a cryptographic key derived from the shared secret embedded in the processor. This signed report (referred to as \textit{quote} \cite{EnclaveWritersGuide}) can then be provided 
to the challenger system which in turn can verify the authenticity through Intel which has access to the shared secret of the target systems processor. If the quote is authentic, 
the challenger system can analyze the measurement of the target enclave embedded in the quote to make sure, the target enclave is running the correct software.
