\section{Sealing}
Once an enclave is destroyed, all contained secrets are gone as well. Intel \ac{sgx} thus enables enclaves to seal data for persistent storage in a secure fashion. In the abstract a
secret unique to the platform (embedded in the processor) and an enclave's signature are used to generate the cryptographic material which is then used to derive encryption keys
for encrypting the data to be stored. Intel distinguishes between two main different sealing policies (\cite{EnclaveWritersGuide}, \cite{IntelSealing}):
\begin{enumerate}
    \item Sealing to one particular enclave: Using an enclave's measurement for the key derivation, the sealed data can only be recovered by another enclave with an identical
          measurement and thus identical structure. Even a different build or version of the same enclave will result in a different enclave measurement and prevent the enclave
          from unsealing the data again.
    \item Sealing to enclaves with the same signing identity: When requiring less strict access to the sealed data, instead of the enclave's measurement, other fields in its
          signature are used to derive the sealing key. Included are the author's identity, the product ID and security version numbers. This enables enclaves by the same
          author and the same product to share sealed secrets. It also means that when updating enclave code, the sealed data does not have to be unsealed by the old version
          and resealed by the new version, providing greater flexibility to the developer.
\end{enumerate}
In both cases, the sealing key is obtained by the EGETKEY instruction which has access to the enclave's \ac{secs} and the secret stored inside the processor. The key derivation material
is hereby fed to an AES-CMAC which provides a 128-bit symmetric key \cite{Costan2016IntelSE}. Since the key derivation process includes a secret embedded in the processor, it is
impossible to reproduce the same key on a different platform even if the key derivation material is known.
